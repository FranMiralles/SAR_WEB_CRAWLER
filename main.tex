\documentclass[12pt,a4paper]{article}
\usepackage[utf8]{inputenc}
\usepackage[T1]{fontenc}
\usepackage[spanish]{babel}
\usepackage{amsmath}
\usepackage{amsfonts}
\usepackage{amssymb}
\usepackage{graphicx}
\usepackage{titlesec}
\usepackage[left=2cm,right=2cm,top=2cm,bottom=2cm]{geometry}
\usepackage{titlesec}
\setcounter{secnumdepth}{4}
\setcounter{tocdepth}{4}
\usepackage{hyperref}

\titleclass{\subsubsubsection}{straight}[\subsection]

\newcounter{subsubsubsection}[subsubsection]
\renewcommand\thesubsubsubsection{\thesubsubsection.\arabic{subsubsubsection}}
\renewcommand\theparagraph{\thesubsubsubsection.\arabic{paragraph}} % optional; useful if paragraphs are to be numbered

\titleformat{\subsubsubsection}
  {\normalfont\normalsize\bfseries}{\thesubsubsubsection}{1em}{}
\titlespacing*{\subsubsubsection}
{0pt}{3.25ex plus 1ex minus .2ex}{1.5ex plus .2ex}

\makeatletter
\renewcommand\paragraph{\@startsection{paragraph}{5}{\z@}%
  {3.25ex \@plus1ex \@minus.2ex}%
  {-1em}%
  {\normalfont\normalsize\bfseries}}
\renewcommand\subparagraph{\@startsection{subparagraph}{6}{\parindent}%
  {3.25ex \@plus1ex \@minus .2ex}%
  {-1em}%
  {\normalfont\normalsize\bfseries}}
\def\toclevel@subsubsubsection{4}
\def\toclevel@paragraph{5}
\def\toclevel@paragraph{6}
\def\l@subsubsubsection{\@dottedtocline{4}{7em}{4em}}
\def\l@paragraph{\@dottedtocline{5}{10em}{5em}}
\def\l@subparagraph{\@dottedtocline{6}{14em}{6em}}
\makeatother

\setcounter{secnumdepth}{4}
\setcounter{tocdepth}{4}

\usepackage{titling}
\title{\textbf{Sistemas de Almacenamiento y\\ Recuperación de Información\\Práctica 4: Buscador web}}
\author{Roberto Ortiz Sanz \\ Francisco Miralles Ferrer \\ Carlos Ruiz Aguirre \\ David Esteso Calatrava}


\begin{document}

\begin{titlepage}
\maketitle
\begin{center}

\vspace{10mm}

\includegraphics[width=0.4\textwidth]{logo_upv.jpg}

\vfill

\large
ETSINF\\
Grado en Ingeniería Informática
\end{center}
\end{titlepage}

\tableofcontents

\newpage

\section{SAR\_Crawler\_Lib.py}
\subsection{parse\_wikipedia\_textual\_content()}
El método \texttt{parse\_wikipedia\_textual\_content} convierte el texto crudo de un artículo de Wikipedia en una estructura organizada y fácil de manejar. A continuación, se describen los pasos y componentes clave del método:

Primero, se definen dos argumentos para el método: \texttt{text}, que es el texto en crudo del artículo de Wikipedia, y \texttt{url}, que es la URL del artículo que se añade como un campo en el resultado.

El objetivo del método es transformar el texto crudo en un diccionario que contenga la URL del artículo, el título, un resumen y una lista de secciones del artículo, donde cada sección puede tener subsecciones.

El proceso comienza con la definición de una función \texttt{clean\_text} que elimina líneas vacías del texto. Luego, se utiliza una expresión regular para intentar extraer el título y el resumen del texto. Si no encuentra título ni resumen, el método devuelve \texttt{None}.

A continuación, se inicializa un diccionario con los campos \texttt{url}, \texttt{title}, \texttt{summary} y \texttt{sections} (inicialmente vacío). Luego, se encuentran las secciones en el texto restante utilizando expresiones regulares. Para cada sección, se identifica el nombre y el contenido, y se crea un diccionario que incluye su nombre, contenido y una lista de subsecciones (inicialmente vacía).

Dentro de cada sección, se encuentran las subsecciones. Para cada subsección, se identifica el nombre y el contenido, se crea un diccionario para la subsección y se añade a la lista de subsecciones de la sección correspondiente.

Finalmente, se añaden las secciones (con sus subsecciones) al diccionario del artículo. El método devuelve un diccionario estructurado con la información del artículo, o \texttt{None} si no se pudo extraer título ni resumen.

\subsection{start\_crawling()}
El método start\_crawling captura contenido de Wikipedia desde una lista inicial de URLs hasta alcanzar un número máximo de documentos o agotar las URLs. Utiliza estructuras de datos para rastrear URLs visitadas, pendientes y en proceso. Descarga y analiza el contenido de cada URL, añadiendo nuevos enlaces válidos para procesar. El contenido textual descargado se analiza mediante una llamada a self.parse\_wikipedia\_textual\_content, que genera un diccionario estructurado del artículo. Los documentos capturados se guardan periódicamente en lotes si se especifica un tamaño de lote, o al finalizar el proceso. Este enfoque la captura de los datos, respetando los límites de documentos y profundidad establecidos.


\section{SAR\_lib.py}
\subsection{Parte 1: Indexación}
\subsubsection{index\_dir()}
Se ha editado el método index\_dir para que permita la generación de los índices permuterm y stemming si se pide en la ejecución del indexador.

\subsubsection{index\_file()}
Para \textbf{indexar el contenido de un fichero} o documento el \textbf{primer paso es asignarle un identificador único}, este corresponderá al nº de documentos ya almacenados en ese momento, así conseguimos que el primer documento tenga un \textit{docID}=0, el segundo tenga un \textit{docID}=1... 

En resumen el \textit{docID} del documento n-ésimo analizado será n-1.\\

\textbf{A continuación} trataremos cada linea del documento como un artículo distinto, pues en cada linea encontramos un artículo en formato \href{https://en.wikipedia.org/wiki/JSON}{JSON}.

Es prioritario tener en cuenta antes de la indexación si el documento ya ha sido indexado, por lo que mediante el método \textbf{self.already\_in\_index()} lo podemos verificar para no considerar este documento.

Tras esto asignamos un \textit{artID} de la misma forma que para cada documento; considerando la cantidad de artículos indexados.\\

Es en este momento cuando empieza el proceso de indexación:

\begin{enumerate}
  \item Se \textit{tokeniza} el contenido del campo en cuestión (el conjunto de campos considerados para \textit{tokenización} depende de si la opción \textit{multifield} está activa).
  \item Para cada \textit{token} \textbf{se añade} en la posting list correspondiente \textbf{el artID} en el que ha aparecido o una tupla (\textbf{artID, pos}) con el artID y la posición que ocupa el token dentro del texto dependiendo de si la opción de índice posicional está activa. \\ \\
  \textbf{Antes de añadir una nueva entrada} a la posting list \textbf{se comprueba si}, en el caso de que la indexación no sea posicional, \textbf{este token ya ha aparecido} en el documento actual para evitar añadir un duplicado. \textbf{De la misma forma si} la indexación es posicional y \textbf{un token ya ha aparecido} en un documento se tiene en cuenta y no se añade la tupla, sino que \textbf{se añade la posición dentro del segundo elemento de la tupla} para almacenar todas las posiciones en las que un token aparece dentro de un documento
  \item Finalmente se añade la URL del artículo al conjunto de URLs, gracias a este conjunto somos capaces de determinar si un artículo ya ha sido indexado
\end{enumerate}

\subsubsection{make\_stemming()}
A la hora de crear el índice de \textit{stemming} para los términos de todos los índices debemos de iterar por todos los campos indexados (depende de la opción multifield) para extraer su \textit{stem} y almacenar todos aquellos términos que comparten stem en el diccionario self.sindex.

Ejemplo: Contenido de self.sindex['all']['program'] tras la indexación de la carpeta '200' (de sus ficheros:
\newline
['programación', 'programa', 'programadora', 'programas', 'programar', 'programador', 'programadores', 'programarlas', 'programadas', 'programada', 'programables', 'programando', 'program', 'programable', 'programarse', 'programados', 'programó', 'programado', 'programaban', 'programan', 'programe', 'programacion']


\subsubsection{make\_permuterm()}
A la hora de crear el índice de \textit{permuterms} para los términos de todos los índices debemos de iterar por todos los campos indexados (depende de la opción multifield) y calcular el índice de permuterms del término con las rotaciones de cada subcadena con el símbolo '\$' de la forma vista en teoría. Deberemos almacenar el término generados por su permuterm en self.ptindex.

Ejemplo: Contenido de self.ptindex['all']['sa\$ca'] tras la indexación de la carpeta '200' (de sus ficheros):
\newline
casa



\subsubsection{show\_stats()}
Para este apartado debemos mostrar la longitud de los distintos diccionarios generados durante el proceso de indexación, así como el número de archivos y artículos indexados. Debe tener este formato:

\begin{verbatim}
    ========================================
    Number of indexed files: 3
    ----------------------------------------
    Number of indexed articles: 589
    ----------------------------------------
    TOKENS:
            # of tokens in 'all': 103358
            # of tokens in 'title': 638
            # of tokens in 'summary': 11474
            # of tokens in 'section-name': 3397
            # of tokens in 'url': 589
    ----------------------------------------
    PERMUTERMS:
            # of permuterms in 'all': 888793
            # of permuterms in 'title': 5095
            # of permuterms in 'summary': 101524
            # of permuterms in 'section-name': 30396
            # of permuterms in 'url': 28155
    ----------------------------------------
    STEMS:
            # of stems in 'all': 71944
            # of stems in 'title': 606
            # of stems in 'summary': 7130
            # of stems in 'section-name': 2614
            # of stems in 'url': 588
    ----------------------------------------
    Positional queries are allowed
    ========================================
    Time indexing: 3.67s.
    Time saving: 0.84s.
\end{verbatim}

\begin{itemize}
  \item La sección PERMUTERMS solo se muestra si se ha realizado el cálculo de \textit{permuterms}
  \item La sección STEMS solo se muestra si se ha realizado el cálculo de \textit{stems}
  \item En una sección se muestran los campos 'all', 'title', 'summary', 'section-name', 'url' si se ha activado la opción de indexación \textit{multifield}, en caso de no haberse activado solo se realiza el cálculo de 'all' y se muestra el campo.
  \item Además se indica si se admiten consultas posicionales, esto varía en función de si se ha realizado la indexación con índices posicionales. 
\end{itemize}

\subsection{Parte 2.1: Recuperación}
\subsubsection{solve\_query()}
El método solve\_query resuelve una consulta desglosando y evaluando términos y operadores. Primero, normaliza la consulta con normalize\_query, convirtiéndola en una lista de tokens, para después obtener las posting lists de los términos y las almacenarlas junto con los operadores en una lista de tokens.
Después aplica el siguiente algoritmo: se utilizan dos pilas, una para operadores y otra para operandos, y recorre la lista de tokens de la consulta para evaluarla. Para darle prioridad a los paréntesis, cuando se encuentra un paréntesis de cierre, el algoritmo procesa los operadores y operandos hasta que encuentra el paréntesis de apertura correspondiente, evaluando así la subexpresión. Esto asegura que las subexpresiones entre paréntesis se evalúen antes que otras partes de la consulta. Para los operadores \textit{AND} y \textit{OR}, evalúa los operadores que se habían almacenado previamente en la pila antes de añadir el nuevo operador, ya que leemos las consultas de izquierda a derecha. Posteriormente, añade el operador a la pila. Si el operador \textit{NOT} es seguido por una subexpresión entre paréntesis, lo añade a la pila; de lo contrario, aplica el operador al siguiente operando. Los términos se añaden directamente a la pila de operandos como sus posting lists correspondientes. El proceso continúa hasta que todos los tokens se procesan y cualquier operador restante se evalúa, obteniendo el resultado final de la pila de operandos. 

\subsubsection{normalize\_query()}
El objetivo del método normalize\_query es convertir la consulta en una lista compuesta de operadores y operandos. Primero, se recorre la consulta para separarla en palabras, a menos que se encuentren dentro de comillas dobles, indicando una consulta posicional, en cuyo caso los espacios dentro de las comillas no separan las palabras. Además, se asegura de que los paréntesis estén correctamente separados de los operandos en la consulta normalizada. Posteriormente, se recorre nuevamente la consulta ya normalizada para insertar operadores lógicos \textit{AND} entre los operandos cuando el usuario no los haya incluido explícitamente. Esto garantiza que la consulta esté estructurada de manera clara y que los operadores lógicos estén presentes donde sean necesarios.

\subsubsection{get\_posting()}
Obtiene una lista de artículos relacionados con los parámetros de entrada del método.
Los parámetros son: el término que se consulta y el campo que indica el diccionario indexado en el que buscar, siendo opcional (si no se introduce, se busca en el campo \textit{all}).

Dependiendo del término de entrada, se hacen llamadas a los métodos:
\begin{itemize}
    \item get\_positionals(): si el término es un término posicional (contiene espacios).
    \item get\_stemming(): si se ha indexado con stemming.
    \item get\_permuterm(): si el término contiene comodines.
\end{itemize}

En cualquier otro caso, se obtiene la \textit{posting list} mediante el diccionario \textit{index}, devolviendo una lista vacía si el término no ha sido indexado.

Por último, el método elimina las posiciones de las \textit{posting list} en el caso de que hayan, ya que solo debe devolver los artículos.

El método devuelve la \textit{posting list} relacionada con el término y campo.

\subsubsection{get\_positionals()}
Devuelve una \textit{posting list} con aquellos artículos que contienen el sintagma pasado como parámetro término y relacionado con el campo.

Para ello, se hace una primera búsqueda de los artículos comunes para todos los términos del sintagma, para esto, se realiza \textit{and} de las \textit{postings lists}. Esto agiliza la computación del bucle posterior.

Una vez teniendo la lista de artículos comunes para todos los términos, se obtiene la referencia de la \textit{posting list} del primer término y se recorre hasta encontrar un elemento cuyo artículo sea el mismo que alguno de la lista de artículos comunes. Al encontrar alguno, se hace una búsqueda a través de sus posiciones y se busca si los demás términos en el mismo artículo se encuentran en una posición sucesora. Aquellas iteraciones que acaben, que hayan recorrido todos los términos y hayan encontrado posición sucesora, almacenarán su artículo en el resultado a devolver.
\subsubsection{get\_stemming()}
Este método devuelve la \textit{posting list} relacionada con el término y campo en el diccionario \textit{sindex}, que almacena los \textit{stemmings} de los términos.
\subsubsection{get\_permuterm()}
El método recibe un término con un comodín \texttt{"}*\texttt{"} o \texttt{"}?\texttt{"} y el campo donde buscar.

 El primer paso es realizar la conversión a \textit{wild card query}, añadiendo un \texttt{"}\$\texttt{"} al final del término y permutando su posición hasta dejar el comodín a la izquierda.

 Una vez realizado, se obtienen todas las claves del índice permuterm \textit{ptindex} y se hace matching de cada clave con el término teniendo en cuenta que:
 \begin{itemize}
\item Si el comodín es \texttt{"}*\texttt{"}, el matching se realiza si la clave contiene como prefijo al término.
\item Si el comodín es \texttt{"}?\texttt{"}, el matching se realiza si la clave contiene como prefijo al término y el sufijo que queda es un único término, ya que el comodín \texttt{"}?\texttt{"} únicamente sustituye a un carácter.

Una vez se tienen todas las claves que hacen matching, se encuentran las palabras relacionadas con las \textit{postings lings} obteniendo el valor de la clave de \textit{ptindex} con las claves. Por último, se realiza el \textit{AND} de todas las \textit{postings lists} y se devuelve.
\end{itemize}

\subsubsection{reverse\_posting()}
Como parámetros contiene una lista de artículos, la cual se quiere conseguir su negado respecto a todo el conjunto de artículos indexados.

Se utiliza para realizar las consultas con el operando \textit{NOT}. Para conseguir esta funcionalidad, se obtienen todos los id de los artículos mediante el diccionario \textit{articles}. Se ordenan y se devuelve el resultado de la operación minus\_posting() explicada en los siguientes apartados con los parámetros de todas las claves y la lista pasada por parámetros. Esto devolverá aquellos artículos que no se encuentran en la lista pasada por parámetros.

\subsubsection{and\_posting()}
Se le pasan por parámetros dos listas las cuales se les aplica un algoritmo que da funcionalidad a las consultas con el operando \textit{AND}.

Las listas se ordenan y se eliminan duplicados, para impedir posibles errores ya que el algoritmo es funcional solo si las listas se encuentran ordenadas ascendentemente.

Se hace un recorrido de ambas listas hasta que alguna de las dos se vacíe. En cada iteración se recupera el primer elemento de las listas. Si este coincide, se añade al resultado y avanzan ambas listas. En caso de no coincidir, avanza la lista cuya elemento sea menor.

\subsubsection{or\_posting()}
Similar al algoritmo de and\_posting() recive dos listas por parámetros, ordenándolos y eliminando repetidos como primer paso.

El recorrido se realiza mientras que ambas listas no estén vacías, obteniendo en cada iteración los primeros elementos de las listas. Si son iguales, se añade uno de los dos y avanzan ambas listas en su recorrido. Si son distintos, se añade el menor elemento y avanza la lista cuyo elemento se ha añadido. Al terminar el bucle, se añaden los elemenos de aquella lista que no está vacía.

\subsubsection{minus\_posting()}

Este método es usado como auxiliar por reverse\_posting() para aportar funcionalidad al operando \textit{NOT}.

Se hace un recorrido hasta que alguna de las listas se vacíe. En cada iteración si el elemento es el mismo en ambas, no se añade ninguno al resultado y se avanza en ambas listas. Si el primer elemento es menor al segundo, se añade el elemento al resultado y se avanza en la lista. Si es el segundo el elemento el menor, solamente se avanza en la lista. Al terminar el bucle, si la longitud de la primera lista aún contiene elementos, se añaden todos estos al resultado.

Con este funcionamiento, se representa la eliminación de los elementos de una lista sobre otra.

\subsection{Parte 2.2: Mostrar resultados}
\subsubsection{solve\_and\_show()}

Y esta va al final del todo:

\subsubsection{Cálculo de Snippets}
Finalmente vamos a exponer como hemos realizado el cálculo de \textit{snippets}, el cual se emplea en el método \textbf{solve\_and\_show() } si se ha activado la opción de mostrar \textit{snippets}.\\
Para realizar el cálculo llamamos a la función \textbf{make\_snippet()} pasando un artículo, una lista de términos y un tamaño de ventana, obtenemos una lista de las ocurrencias de los términos de la consulta y determinamos el inicio y el final del \textit{snippet} empleando la posición mínima y máxima de los índices anteriores asegurándonos de que no exceda los límites del texto y que tenga el tamaño de ventana especificado (por defecto 50 unidades).

\end{document}
